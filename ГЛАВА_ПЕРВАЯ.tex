%!TEX program = xelatex
\documentclass[12pt,twocolumn]{article}
%\usepackage{extsizes}
\usepackage[margin=1.0in,letterpaper]{geometry}
\usepackage[T2A,T1]{fontenc}
\usepackage{fontspec}
\usepackage[russian,english]{babel}
\usepackage{verse}
\usepackage{alltt}
\usepackage{setspace}

\setmainfont{Arial}

\linespread{2.5}

\begin{document}
\begin{center}
\begin{otherlanguage*}{russian}
\begin{minipage}[t]{\dimexpr 0.5\textwidth -\tabcolsep-.5pt}
\begin{alltt}\normalfont\centering
I
«Мой дядя самых честных правил,
Когда не в шутку занемог,
Он уважать себя заставил
И лучше выдумать не мог.
Его пример другим наука;
Но, боже мой, какая скука
С больным сидеть и день и ночь,
Не отходя ни шагу прочь!
Какое низкое коварство
Полуживого забавлять,
Ему подушки поправлять,
Печально подносить лекарство,
Вздыхать и думать про себя:
Когда же черт возьмет тебя!»
\end{alltt}
\end{minipage}

\begin{minipage}[t]{\dimexpr 0.5\textwidth -\tabcolsep-.5pt}
\begin{alltt}\normalfont\centering
II
Так думал молодой повеса,
Летя в пыли на почтовых,
Всевышней волею Зевеса
Наследник всех своих родных.
Друзья Людмилы и Руслана!
С героем моего романа
Без предисловий, сей же час
Позвольте познакомить вас:
Онегин, добрый мой приятель,
Родился на брегах Невы,
Где, может быть, родились вы
Или блистали, мой читатель;
Там некогда гулял и я:
Но вреден север для меня.
\end{alltt}
\end{minipage}
\clearpage

\begin{minipage}[t]{\dimexpr 0.5\textwidth -\tabcolsep-.5pt}
\begin{alltt}\normalfont\centering
III
Служив отлично благородно,
Долгами жил его отец,
Давал три бала ежегодно
И промотался наконец.
Судьба Евгения хранила:
Сперва Madame за ним ходила,
Потом Monsieur ее сменил.
Ребенок был резов, но мил.
Monsieur l'Abbé, француз убогой,
Чтоб не измучилось дитя,
Учил его всему шутя,
Не докучал моралью строгой,
Слегка за шалости бранил
И в Летний сад гулять водил.
\end{alltt}
\end{minipage}

\begin{minipage}[t]{\dimexpr 0.5\textwidth -\tabcolsep-.5pt}
\begin{alltt}\normalfont\centering
IV
Когда же юности мятежной
Пришла Евгению пора,
Пора надежд и грусти нежной,
Monsieur прогнали со двора.
Вот мой Онегин на свободе;
Острижен по последней моде,
Как dandy лондонский одет —
И наконец увидел свет.
Он по-французски совершенно
Мог изъясняться и писал;
Легко мазурку танцевал
И кланялся непринужденно;
Чего ж вам больше? Свет решил,
Что он умен и очень мил.
\end{alltt}
\end{minipage}
\clearpage

\begin{minipage}[t]{\dimexpr 0.5\textwidth -\tabcolsep-.5pt}
\begin{alltt}\normalfont\centering
V
Мы все учились понемногу
Чему-нибудь и как-нибудь,
Так воспитаньем, слава богу,
У нас немудрено блеснуть.
Онегин был по мненью многих
(Судей решительных и строгих)
Ученый малый, но педант:
Имел он счастливый талант
Без принужденья в разговоре
Коснуться до всего слегка,
С ученым видом знатока
Хранить молчанье в важном споре
И возбуждать улыбку дам
Огнем нежданных эпиграмм.
\end{alltt}
\end{minipage}

\begin{minipage}[t]{\dimexpr 0.5\textwidth -\tabcolsep-.5pt}
\begin{alltt}\normalfont\centering
VI
Латынь из моды вышла ныне:
Так, если правду вам сказать,
Он знал довольно по-латыне,
Чтоб эпиграфы разбирать,
Потолковать об Ювенале,
В конце письма поставить vale,
Да помнил, хоть не без греха,
Из Энеиды два стиха.
Он рыться не имел охоты
В хронологической пыли
Бытописания земли:
Но дней минувших анекдоты
От Ромула до наших дней
Хранил он в памяти своей.
\end{alltt}
\end{minipage}
\clearpage

\begin{minipage}[t]{\dimexpr 0.5\textwidth -\tabcolsep-.5pt}
\begin{alltt}\normalfont\centering
VII
Высокой страсти не имея
Для звуков жизни не щадить,
Не мог он ямба от хорея,
Как мы ни бились, отличить.
Бранил Гомера, Феокрита;
Зато читал Адама Смита
И был глубокой эконом,
То есть умел судить о том,
Как государство богатеет,
И чем живет, и почему
Не нужно золота ему,
Когда простой продукт имеет.
Отец понять его не мог
И земли отдавал в залог.
\end{alltt}
\end{minipage}

\begin{minipage}[t]{\dimexpr 0.5\textwidth -\tabcolsep-.5pt}
\begin{alltt}\normalfont\centering
VIII
Всего, что знал еще Евгений,
Пересказать мне недосуг;
Но в чем он истинный был гений,
Что знал он тверже всех наук,
Что было для него измлада
И труд, и мука, и отрада,
Что занимало целый день
Его тоскующую лень, —
Была наука страсти нежной,
Которую воспел Назон,
За что страдальцем кончил он
Свой век блестящий и мятежный
В Молдавии, в глуши степей,
Вдали Италии своей.
\end{alltt}
\end{minipage}
\clearpage

\begin{minipage}[t]{\dimexpr 0.5\textwidth -\tabcolsep-.5pt}
\begin{alltt}\normalfont\centering
IX
 ........................................................
 ........................................................
 ........................................................
\end{alltt}
\end{minipage}

\vspace{4in}

\begin{minipage}[t]{\dimexpr 0.5\textwidth -\tabcolsep-.5pt}
\begin{alltt}\normalfont\centering
X
Как рано мог он лицемерить,
Таить надежду, ревновать,
Разуверять, заставить верить,
Казаться мрачным, изнывать,
Являться гордым и послушным,
Внимательным иль равнодушным!
Как томно был он молчалив,
Как пламенно красноречив,
В сердечных письмах как небрежен!
Одним дыша, одно любя,
Как он умел забыть себя!
Как взор его был быстр и нежен,
Стыдлив и дерзок, а порой
Блистал послушною слезой!
\end{alltt}
\end{minipage}
\clearpage

\begin{minipage}[t]{\dimexpr 0.5\textwidth -\tabcolsep-.5pt}
\begin{alltt}\normalfont\centering
XI
Как он умел казаться новым,
Шутя невинность изумлять,
Пугать отчаяньем готовым,
Приятной лестью забавлять,
Ловить минуту умиленья,
Невинных лет предубежденья
Умом и страстью побеждать,
Невольной ласки ожидать,
Молить и требовать признанья,
Подслушать сердца первый звук,
Преследовать любовь, и вдруг
Добиться тайного свиданья...
И после ей наедине
Давать уроки в тишине!
\end{alltt}
\end{minipage}

\begin{minipage}[t]{\dimexpr 0.5\textwidth -\tabcolsep-.5pt}
\begin{alltt}\normalfont\centering
XII
Как рано мог уж он тревожить
Сердца кокеток записных!
Когда ж хотелось уничтожить
Ему соперников своих,
Как он язвительно злословил!
Какие сети им готовил!
Но вы, блаженные мужья,
С ним оставались вы друзья:
Его ласкал супруг лукавый,
Фобласа давний ученик,
И недоверчивый старик,
И рогоносец величавый,
Всегда довольный сам собой,
Своим обедом и женой.
\end{alltt}
\end{minipage}
\clearpage

\begin{minipage}[t]{\dimexpr 0.5\textwidth -\tabcolsep-.5pt}
\begin{alltt}\normalfont\centering
XIII. XIV
 ........................................................
 ........................................................
 ........................................................
\end{alltt}
\end{minipage}
\vspace{4in}
\clearpage


\begin{minipage}[t]{\dimexpr 0.5\textwidth -\tabcolsep-.5pt}
\begin{alltt}\normalfont\centering
XV
Бывало, он еще в постеле:
К нему записочки несут.
Что? Приглашенья? В самом деле,
Три дома на вечер зовут:
Там будет бал, там детский праздник.
Куда ж поскачет мой проказник?
С кого начнет он? Все равно:
Везде поспеть немудрено.
Покамест в утреннем уборе,
Надев широкий боливар 3,
Онегин едет на бульвар
И там гуляет на просторе,
Пока недремлющий брегет
Не прозвонит ему обед.
\end{alltt}
\end{minipage}

\begin{minipage}[t]{\dimexpr 0.5\textwidth -\tabcolsep-.5pt}
\begin{alltt}\normalfont\centering
XVI
Уж тёмно: в санки он садится.
«Пади, пади!» — раздался крик;
Морозной пылью серебрится
Его бобровый воротник.
К Talon 4 помчался: он уверен,
Что там уж ждет его Каверин.
Вошел: и пробка в потолок,
Вина кометы брызнул ток;
Пред ним roast-beef окровавленный,
И трюфли, роскошь юных лет,
Французской кухни лучший цвет,
И Страсбурга пирог нетленный
Меж сыром лимбургским живым
И ананасом золотым.
\end{alltt}
\end{minipage}
\clearpage

\begin{minipage}[t]{\dimexpr 0.5\textwidth -\tabcolsep-.5pt}
\begin{alltt}\normalfont\centering
XVII
Еще бокалов жажда просит
Залить горячий жир котлет,
Но звон брегета им доносит,
Что новый начался балет.
Театра злой законодатель,
Непостоянный обожатель
Очаровательных актрис,
Почетный гражданин кулис,
Онегин полетел к театру,
Где каждый, вольностью дыша,
Готов охлопать entrechat,
Обшикать Федру, Клеопатру,
Моину вызвать (для того,
Чтоб только слышали его).
\end{alltt}
\end{minipage}

\begin{minipage}[t]{\dimexpr 0.5\textwidth -\tabcolsep-.5pt}
\begin{alltt}\normalfont\centering
XVIII
Волшебный край! там в стары годы,
Сатиры смелый властелин,
Блистал Фонвизин, друг свободы,
И переимчивый Княжнин;
Там Озеров невольны дани
Народных слез, рукоплесканий
С младой Семеновой делил;
Там наш Катенин воскресил
Корнеля гений величавый;
Там вывел колкий Шаховской
Своих комедий шумный рой,
Там и Дидло венчался славой,
Там, там под сению кулис
Младые дни мои неслись.
\end{alltt}
\end{minipage}
\clearpage

\begin{minipage}[t]{\dimexpr 0.5\textwidth -\tabcolsep-.5pt}
\begin{alltt}\normalfont\centering
XIX
Мои богини! что вы? где вы?
Внемлите мой печальный глас:
Всё те же ль вы? другие ль девы,
Сменив, не заменили вас?
Услышу ль вновь я ваши хоры?
Узрю ли русской Терпсихоры
Душой исполненный полет?
Иль взор унылый не найдет
Знакомых лиц на сцене скучной,
И, устремив на чуждый свет
Разочарованный лорнет,
Веселья зритель равнодушный,
Безмолвно буду я зевать
И о былом воспоминать?
\end{alltt}
\end{minipage}

\begin{minipage}[t]{\dimexpr 0.5\textwidth -\tabcolsep-.5pt}
\begin{alltt}\normalfont\centering
XX
Театр уж полон; ложи блещут;
Партер и кресла — все кипит;
В райке нетерпеливо плещут,
И, взвившись, занавес шумит.
Блистательна, полувоздушна,
Смычку волшебному послушна,
Толпою нимф окружена,
Стоит Истомина; она,
Одной ногой касаясь пола,
Другою медленно кружит,
И вдруг прыжок, и вдруг летит,
Летит, как пух от уст Эола;
То стан совьет, то разовьет
И быстрой ножкой ножку бьет.
\end{alltt}
\end{minipage}

\end{otherlanguage*}
\end{center}
\end{document}