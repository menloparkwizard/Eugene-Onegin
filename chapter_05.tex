\begin{minipage}[t]{\dimexpr 0.5\textwidth -\tabcolsep-.5pt}
\begin{alltt}\normalfont\centering
I

В тот год осенняя погода
Стояла долго на дворе,
Зимы ждала, ждала природа.
Снег выпал только в январе
На третье в ночь. Проснувшись рано,
В окно увидела Татьяна
Поутру побелевший двор,
Куртины, кровли и забор,
На стеклах легкие узоры,
Деревья в зимнем серебре,
Сорок веселых на дворе
И мягко устланные горы
Зимы блистательным ковром.
Все ярко, все бело кругом.
\end{alltt}
\end{minipage}

\begin{minipage}[t]{\dimexpr 0.5\textwidth -\tabcolsep-.5pt}
\begin{alltt}\normalfont\centering
II

Зима!.. Крестьянин, торжествуя,
На дровнях обновляет путь;
Его лошадка, снег почуя,
Плетется рысью как-нибудь;
Бразды пушистые взрывая,
Летит кибитка удалая;
Ямщик сидит на облучке
В тулупе, в красном кушаке.
Вот бегает дворовый мальчик,
В салазки жучку посадив,
Себя в коня преобразив;
Шалун уж заморозил пальчик:
Ему и больно и смешно,
А мать грозит ему в окно...
\end{alltt}
\end{minipage}
\clearpage

\begin{minipage}[t]{\dimexpr 0.5\textwidth -\tabcolsep-.5pt}
\begin{alltt}\normalfont\centering
III

Но, может быть, такого рода
Картины вас не привлекут:
Все это низкая природа;
Изящного не много тут.
Согретый вдохновенья богом,
Другой поэт роскошным слогом
Живописал нам первый снег
И все оттенки зимних нег; 27
Он вас пленит, я в том уверен,
Рисуя в пламенных стихах
Прогулки тайные в санях;
Но я бороться не намерен
Ни с ним покамест, ни с тобой,
Певец финляндки молодой! 28
\end{alltt}
\end{minipage}

\begin{minipage}[t]{\dimexpr 0.5\textwidth -\tabcolsep-.5pt}
\begin{alltt}\normalfont\centering
IV

Татьяна (русская душою,
Сама не зная почему)
С ее холодною красою
Любила русскую зиму,
На солнце иний в день морозный,
И сани, и зарею поздной
Сиянье розовых снегов,
И мглу крещенских вечеров.
По старине торжествовали
В их доме эти вечера:
Служанки со всего двора
Про барышень своих гадали
И им сулили каждый год
Мужьев военных и поход.
\end{alltt}
\end{minipage}
\clearpage

\begin{minipage}[t]{\dimexpr 0.5\textwidth -\tabcolsep-.5pt}
\begin{alltt}\normalfont\centering
V

Татьяна верила преданьям
Простонародной старины,
И снам, и карточным гаданьям,
И предсказаниям луны.
Ее тревожили приметы;
Таинственно ей все предметы
Провозглашали что-нибудь,
Предчувствия теснили грудь.
Жеманный кот, на печке сидя,
Мурлыча, лапкой рыльце мыл:
То несомненный знак ей был,
Что едут гости. Вдруг увидя
Младой двурогий лик луны
На небе с левой стороны,
\end{alltt}
\end{minipage}

\begin{minipage}[t]{\dimexpr 0.5\textwidth -\tabcolsep-.5pt}
\begin{alltt}\normalfont\centering
VI

Она дрожала и бледнела.
Когда ж падучая звезда
По небу темному летела
И рассыпалася, — тогда
В смятенье Таня торопилась,
Пока звезда еще катилась,
Желанье сердца ей шепнуть.
Когда случалось где-нибудь
Ей встретить черного монаха
Иль быстрый заяц меж полей
Перебегал дорогу ей,
Не зная, что начать со страха,
Предчувствий горестных полна,
Ждала несчастья уж она.
\end{alltt}
\end{minipage}
\clearpage

\begin{minipage}[t]{\dimexpr 0.5\textwidth -\tabcolsep-.5pt}
\begin{alltt}\normalfont\centering
VII

Что ж? Тайну прелесть находила
И в самом ужасе она:
Так нас природа сотворила,
К противуречию склонна.
Настали святки. То-то радость!
Гадает ветреная младость,
Которой ничего не жаль,
Перед которой жизни даль
Лежит светла, необозрима;
Гадает старость сквозь очки
У гробовой своей доски,
Все потеряв невозвратимо;
И все равно: надежда им
Лжет детским лепетом своим.
\end{alltt}
\end{minipage}

\begin{minipage}[t]{\dimexpr 0.5\textwidth -\tabcolsep-.5pt}
\begin{alltt}\normalfont\centering
VIII

Татьяна любопытным взором
На воск потопленный глядит:
Он чудно вылитым узором
Ей что-то чудное гласит;
Из блюда, полного водою,
Выходят кольцы чередою;
И вынулось колечко ей
Под песенку старинных дней:
«Там мужички-то всё богаты,
Гребут лопатой серебро;
Кому поем, тому добро
И слава!» Но сулит утраты
Сей песни жалостный напев;
Милей кошурка сердцу дев 29.
\end{alltt}
\end{minipage}
\clearpage

\begin{minipage}[t]{\dimexpr 0.5\textwidth -\tabcolsep-.5pt}
\begin{alltt}\normalfont\centering
IX

Морозна ночь, все небо ясно;
Светил небесных дивный хор
Течет так тихо, так согласно...
Татьяна на широкой двор
В открытом платьице выходит,
На месяц зеркало наводит;
Но в темном зеркале одна
Дрожит печальная луна...
Чу... снег хрустит... прохожий; дева
К нему на цыпочках летит,
И голосок ее звучит
Нежней свирельного напева:
Как ваше имя? 30 Смотрит он
И отвечает: Агафон.
\end{alltt}
\end{minipage}

\begin{minipage}[t]{\dimexpr 0.5\textwidth -\tabcolsep-.5pt}
\begin{alltt}\normalfont\centering
X

Татьяна, по совету няни
Сбираясь ночью ворожить,
Тихонько приказала в бане
На два прибора стол накрыть;
Но стало страшно вдруг Татьяне...
И я — при мысли о Светлане
Мне стало страшно — так и быть...
С Татьяной нам не ворожить.
Татьяна поясок шелковый
Сняла, разделась и в постель
Легла. Над нею вьется Лель,
А под подушкою пуховой
Девичье зеркало лежит.
Утихло все. Татьяна спит.
\end{alltt}
\end{minipage}
\clearpage

\begin{minipage}[t]{\dimexpr 0.5\textwidth -\tabcolsep-.5pt}
\begin{alltt}\normalfont\centering
XI

И снится чудный сон Татьяне.
Ей снится, будто бы она
Идет по снеговой поляне,
Печальной мглой окружена;
В сугробах снежных перед нею
Шумит, клубит волной своею
Кипучий, темный и седой
Поток, не скованный зимой;
Две жердочки, склеены льдиной,
Дрожащий, гибельный мосток,
Положены через поток;
И пред шумящею пучиной,
Недоумения полна,
Остановилася она.
\end{alltt}
\end{minipage}

\begin{minipage}[t]{\dimexpr 0.5\textwidth -\tabcolsep-.5pt}
\begin{alltt}\normalfont\centering
XII

Как на досадную разлуку,
Татьяна ропщет на ручей;
Не видит никого, кто руку
С той стороны подал бы ей;
Но вдруг сугроб зашевелился.
И кто ж из-под него явился?
Большой, взъерошенный медведь;
Татьяна ах! а он реветь,
И лапу с острыми когтями
Ей протянул; она скрепясь
Дрожащей ручкой оперлась
И боязливыми шагами
Перебралась через ручей;
Пошла — и что ж? медведь за ней!
\end{alltt}
\end{minipage}
\clearpage

\begin{minipage}[t]{\dimexpr 0.5\textwidth -\tabcolsep-.5pt}
\begin{alltt}\normalfont\centering
XIII

Она, взглянуть назад не смея,
Поспешный ускоряет шаг;
Но от косматого лакея
Не может убежать никак;
Кряхтя, валит медведь несносный;
Пред ними лес; недвижны сосны
В своей нахмуренной красе;
Отягчены их ветви все
Клоками снега; сквозь вершины
Осин, берез и лип нагих
Сияет луч светил ночных;
Дороги нет; кусты, стремнины
Метелью все занесены,
Глубоко в снег погружены.
\end{alltt}
\end{minipage}

\begin{minipage}[t]{\dimexpr 0.5\textwidth -\tabcolsep-.5pt}
\begin{alltt}\normalfont\centering
XIV

Татьяна в лес; медведь за нею;
Снег рыхлый по колено ей;
То длинный сук ее за шею
Зацепит вдруг, то из ушей
Златые серьги вырвет силой;
То в хрупком снеге с ножки милой
Увязнет мокрый башмачок;
То выронит она платок;
Поднять ей некогда; боится,
Медведя слышит за собой,
И даже трепетной рукой
Одежды край поднять стыдится;
Она бежит, он все вослед,
И сил уже бежать ей нет.
\end{alltt}
\end{minipage}
\clearpage

\begin{minipage}[t]{\dimexpr 0.5\textwidth -\tabcolsep-.5pt}
\begin{alltt}\normalfont\centering
XV

Упала в снег; медведь проворно
Ее хватает и несет;
Она бесчувственно-покорна,
Не шевельнется, не дохнет;
Он мчит ее лесной дорогой;
Вдруг меж дерев шалаш убогой;
Кругом все глушь; отвсюду он
Пустынным снегом занесен,
И ярко светится окошко,
И в шалаше и крик и шум;
Медведь промолвил: «Здесь мой кум:
Погрейся у него немножко!»
И в сени прямо он идет
И на порог ее кладет.
\end{alltt}
\end{minipage}

\begin{minipage}[t]{\dimexpr 0.5\textwidth -\tabcolsep-.5pt}
\begin{alltt}\normalfont\centering
XVI

Опомнилась, глядит Татьяна:
Медведя нет; она в сенях;
За дверью крик и звон стакана,
Как на больших похоронах;
Не видя тут ни капли толку,
Глядит она тихонько в щелку,
И что же видит?.. за столом
Сидят чудовища кругом:
Один в рогах с собачьей мордой,
Другой с петушьей головой,
Здесь ведьма с козьей бородой,
Тут остов чопорный и гордый,
Там карла с хвостиком, а вот
Полужуравль и полукот.
\end{alltt}
\end{minipage}
\clearpage

\begin{minipage}[t]{\dimexpr 0.5\textwidth -\tabcolsep-.5pt}
\begin{alltt}\normalfont\centering
XVII

Еще страшней, еще чуднее:
Вот рак верхом на пауке,
Вот череп на гусиной шее
Вертится в красном колпаке,
Вот мельница вприсядку пляшет
И крыльями трещит и машет;
Лай, хохот, пенье, свист и хлоп,
Людская молвь и конской топ! 31
Но что подумала Татьяна,
Когда узнала меж гостей
Того, кто мил и страшен ей,
Героя нашего романа!
Онегин за столом сидит
И в дверь украдкою глядит.
\end{alltt}
\end{minipage}

\begin{minipage}[t]{\dimexpr 0.5\textwidth -\tabcolsep-.5pt}
\begin{alltt}\normalfont\centering
XVIII

Он знак подаст — и все хлопочут;
Он пьет — все пьют и все кричат;
Он засмеется — все хохочут;
Нахмурит брови — все молчат;
Он там хозяин, это ясно:
И Тане уж не так ужасно,
И, любопытная, теперь
Немного растворила дверь...
Вдруг ветер дунул, загашая
Огонь светильников ночных;
Смутилась шайка домовых;
Онегин, взорами сверкая,
Из-за стола, гремя, встает;
Все встали: он к дверям идет.
\end{alltt}
\end{minipage}
\clearpage

\begin{minipage}[t]{\dimexpr 0.5\textwidth -\tabcolsep-.5pt}
\begin{alltt}\normalfont\centering
XIX

И страшно ей; и торопливо
Татьяна силится бежать:
Нельзя никак; нетерпеливо
Метаясь, хочет закричать:
Не может; дверь толкнул Евгений:
И взорам адских привидений
Явилась дева; ярый смех
Раздался дико; очи всех,
Копыты, хоботы кривые,
Хвосты хохлатые, клыки,
Усы, кровавы языки,
Рога и пальцы костяные,
Всё указует на нее,
И все кричат: мое! мое!
\end{alltt}
\end{minipage}

\begin{minipage}[t]{\dimexpr 0.5\textwidth -\tabcolsep-.5pt}
\begin{alltt}\normalfont\centering
XX

Мое! — сказал Евгений грозно,
И шайка вся сокрылась вдруг;
Осталася во тьме морозной
Младая дева с ним сам-друг;
Онегин тихо увлекает 32
Татьяну в угол и слагает
Ее на шаткую скамью
И клонит голову свою
К ней на плечо; вдруг Ольга входит,
За нею Ленский; свет блеснул;
Онегин руку замахнул,
И дико он очами бродит,
И незваных гостей бранит;
Татьяна чуть жива лежит.
\end{alltt}
\end{minipage}
\clearpage

\begin{minipage}[t]{\dimexpr 0.5\textwidth -\tabcolsep-.5pt}
\begin{alltt}\normalfont\centering
XXI

Спор громче, громче; вдруг Евгений
Хватает длинный нож, и вмиг
Повержен Ленский; страшно тени
Сгустились; нестерпимый крик
Раздался... хижина шатнулась...
И Таня в ужасе проснулась...
Глядит, уж в комнате светло;
В окне сквозь мерзлое стекло
Зари багряный луч играет;
Дверь отворилась. Ольга к ней,
Авроры северной алей
И легче ласточки, влетает;
«Ну, говорит, скажи ж ты мне,
Кого ты видела во сне?»
\end{alltt}
\end{minipage}

\begin{minipage}[t]{\dimexpr 0.5\textwidth -\tabcolsep-.5pt}
\begin{alltt}\normalfont\centering
XXII

Но та, сестры не замечая,
В постеле с книгою лежит,
За листом лист перебирая,
И ничего не говорит.
Хоть не являла книга эта
Ни сладких вымыслов поэта,
Ни мудрых истин, ни картин,
Но ни Виргилий, ни Расин,
Ни Скотт, ни Байрон, ни Сенека,
Ни даже Дамских Мод Журнал
Так никого не занимал:
То был, друзья, Мартын Задека 33,
Глава халдейских мудрецов,
Гадатель, толкователь снов.
\end{alltt}
\end{minipage}
\clearpage

\begin{minipage}[t]{\dimexpr 0.5\textwidth -\tabcolsep-.5pt}
\begin{alltt}\normalfont\centering
XXIII

Сие глубокое творенье
Завез кочующий купец
Однажды к ним в уединенье
И для Татьяны наконец
Его с разрозненной «Мальвиной»
Он уступил за три с полтиной,
В придачу взяв еще за них
Собранье басен площадных,
Грамматику, две Петриады
Да Мармонтеля третий том.
Мартын Задека стал потом
Любимец Тани... Он отрады
Во всех печалях ей дарит
И безотлучно с нею спит.
\end{alltt}
\end{minipage}

\begin{minipage}[t]{\dimexpr 0.5\textwidth -\tabcolsep-.5pt}
\begin{alltt}\normalfont\centering
XXIV

Ее тревожит сновиденье.
Не зная, как его понять,
Мечтанья страшного значенье
Татьяна хочет отыскать.
Татьяна в оглавленье кратком
Находит азбучным порядком
Слова: бор, буря, ведьма, ель,
Еж, мрак, мосток, медведь, метель
И прочая. Ее сомнений
Мартын Задека не решит;
Но сон зловещий ей сулит
Печальных много приключений.
Дней несколько она потом
Все беспокоилась о том.
\end{alltt}
\end{minipage}
\clearpage

\begin{minipage}[t]{\dimexpr 0.5\textwidth -\tabcolsep-.5pt}
\begin{alltt}\normalfont\centering
XXV

Но вот багряною рукою 34
Заря от утренних долин
Выводит с солнцем за собою
Веселый праздник именин.
С утра дом Лариных гостями
Весь полон; целыми семьями
Соседи съехались в возках,
В кибитках, в бричках и в санях.
В передней толкотня, тревога;
В гостиной встреча новых лиц,
Лай мосек, чмоканье девиц,
Шум, хохот, давка у порога,
Поклоны, шарканье гостей,
Кормилиц крик и плач детей.
\end{alltt}
\end{minipage}

\begin{minipage}[t]{\dimexpr 0.5\textwidth -\tabcolsep-.5pt}
\begin{alltt}\normalfont\centering
XXVI

С своей супругою дородной
Приехал толстый Пустяков;
Гвоздин, хозяин превосходный,
Владелец нищих мужиков;
Скотинины, чета седая,
С детьми всех возрастов, считая
От тридцати до двух годов;
Уездный франтик Петушков,
Мой брат двоюродный, Буянов,
В пуху, в картузе с козырьком 35
(Как вам, конечно, он знаком),
И отставной советник Флянов,
Тяжелый сплетник, старый плут,
Обжора, взяточник и шут.
\end{alltt}
\end{minipage}
\clearpage

\begin{minipage}[t]{\dimexpr 0.5\textwidth -\tabcolsep-.5pt}
\begin{alltt}\normalfont\centering
XXVII

С семьей Панфила Харликова
Приехал и мосье Трике,
Остряк, недавно из Тамбова,
В очках и в рыжем парике.
Как истинный француз, в кармане
Трике привез куплет Татьяне
На голос, знаемый детьми:
Réveillez vous, belle endormie.
Меж ветхих песен альманаха
Был напечатан сей куплет;
Трике, догадливый поэт,
Его на свет явил из праха,
И смело вместо belle Nina
Поставил belle Tatiana.
\end{alltt}
\end{minipage}

\begin{minipage}[t]{\dimexpr 0.5\textwidth -\tabcolsep-.5pt}
\begin{alltt}\normalfont\centering
XXVIII

И вот из ближнего посада
Созревших барышень кумир,
Уездных матушек отрада,
Приехал ротный командир;
Вошел... Ах, новость, да какая!
Музыка будет полковая!
Полковник сам ее послал.
Какая радость: будет бал!
Девчонки прыгают заране; 36
Но кушать подали. Четой
Идут за стол рука с рукой.
Теснятся барышни к Татьяне;
Мужчины против; и, крестясь,
Толпа жужжит, за стол садясь.
\end{alltt}
\end{minipage}
\clearpage

\begin{minipage}[t]{\dimexpr 0.5\textwidth -\tabcolsep-.5pt}
\begin{alltt}\normalfont\centering
XXIX

На миг умолкли разговоры;
Уста жуют. Со всех сторон
Гремят тарелки и приборы
Да рюмок раздается звон.
Но вскоре гости понемногу
Подъемлют общую тревогу.
Никто не слушает, кричат,
Смеются, спорят и пищат.
Вдруг двери настежь. Ленский входит,
И с ним Онегин. «Ах, творец! —
Кричит хозяйка: — наконец!»
Теснятся гости, всяк отводит
Приборы, стулья поскорей;
Зовут, сажают двух друзей.
\end{alltt}
\end{minipage}

\begin{minipage}[t]{\dimexpr 0.5\textwidth -\tabcolsep-.5pt}
\begin{alltt}\normalfont\centering
XXX

Сажают прямо против Тани,
И, утренней луны бледней
И трепетней гонимой лани,
Она темнеющих очей
Не подымает: пышет бурно
В ней страстный жар; ей душно, дурно;
Она приветствий двух друзей
Не слышит, слезы из очей
Хотят уж капать; уж готова
Бедняжка в обморок упасть;
Но воля и рассудка власть
Превозмогли. Она два слова
Сквозь зубы молвила тишком
И усидела за столом.
\end{alltt}
\end{minipage}
\clearpage

\begin{minipage}[t]{\dimexpr 0.5\textwidth -\tabcolsep-.5pt}
\begin{alltt}\normalfont\centering
XXXI

Траги-нервичсских явлений,
Девичьих обмороков, слез
Давно терпеть не мог Евгений:
Довольно их он перенес.
Чудак, попав на пир огромный,
Уж был сердит. Но девы томной
Заметя трепетный порыв,
С досады взоры опустив,
Надулся он и, негодуя,
Поклялся Ленского взбесить
И уж порядком отомстить.
Теперь, заране торжествуя,
Он стал чертить в душе своей
Карикатуры всех гостей.
\end{alltt}
\end{minipage}

\begin{minipage}[t]{\dimexpr 0.5\textwidth -\tabcolsep-.5pt}
\begin{alltt}\normalfont\centering
XXXII

Конечно, не один Евгений
Смятенье Тани видеть мог;
Но целью взоров и суждений
В то время жирный был пирог
(К несчастию, пересоленный);
Да вот в бутылке засмоленной,
Между жарким и блан-манже,
Цимлянское несут уже;
За ним строй рюмок узких, длинных,
Подобно талии твоей,
Зизи, кристалл души моей,
Предмет стихов моих невинных,
Любви приманчивый фиал,
Ты, от кого я пьян бывал!
\end{alltt}
\end{minipage}
\clearpage

\begin{minipage}[t]{\dimexpr 0.5\textwidth -\tabcolsep-.5pt}
\begin{alltt}\normalfont\centering
XXXIII

Освободясь от пробки влажной,
Бутылка хлопнула; вино
Шипит; и вот с осанкой важной,
Куплетом мучимый давно,
Трике встает; пред ним собранье
Хранит глубокое молчанье.
Татьяна чуть жива; Трике,
К ней обратясь с листком в руке,
Запел, фальшивя. Плески, клики
Его приветствуют. Она
Певцу присесть принуждена;
Поэт же скромный, хоть великий,
Ее здоровье первый пьет
И ей куплет передает.
\end{alltt}
\end{minipage}

\begin{minipage}[t]{\dimexpr 0.5\textwidth -\tabcolsep-.5pt}
\begin{alltt}\normalfont\centering
XXXIV

Пошли приветы, поздравленья;
Татьяна всех благодарит.
Когда же дело до Евгенья
Дошло, то девы томный вид,
Ее смущение, усталость
В его душе родили жалость:
Он молча поклонился ей,
Но как-то взор его очей
Был чудно нежен. Оттого ли,
Что он и вправду тронут был,
Иль он, кокетствуя, шалил,
Невольно ль, иль из доброй воли,
Но взор сей нежность изъявил:
Он сердце Тани оживил.
\end{alltt}
\end{minipage}
\clearpage

\begin{minipage}[t]{\dimexpr 0.5\textwidth -\tabcolsep-.5pt}
\begin{alltt}\normalfont\centering
XXXV

Гремят отдвинутые стулья;
Толпа в гостиную валит:
Так пчел из лакомого улья
На ниву шумный рой летит.
Довольный праздничным обедом,
Сосед сопит перед соседом;
Подсели дамы к камельку;
Девицы шепчут в уголку;
Столы зеленые раскрыты:
Зовут задорных игроков
Бостон и ломбер стариков,
И вист, доныне знаменитый,
Однообразная семья,
Все жадной скуки сыновья.
\end{alltt}
\end{minipage}

\begin{minipage}[t]{\dimexpr 0.5\textwidth -\tabcolsep-.5pt}
\begin{alltt}\normalfont\centering
XXXVI

Уж восемь робертов сыграли
Герои виста; восемь раз
Они места переменяли;
И чай несут. Люблю я час
Определять обедом, чаем
И ужином. Мы время знаем
В деревне без больших сует:
Желудок — верный наш брегет;
И кстати я замечу в скобках,
Что речь веду в моих строфах
Я столь же часто о пирах,
О разных кушаньях и пробках,
Как ты, божественный Омир,
Ты, тридцати веков кумир!
\end{alltt}
\end{minipage}
\clearpage

\begin{minipage}[t]{\dimexpr 0.5\textwidth -\tabcolsep-.5pt}
\begin{alltt}\normalfont\centering
XXXVII. XXXVIII. XXXIX

Но чай несут; девицы чинно
Едва за блюдички взялись,
Вдруг из-за двери в зале длинной
Фагот и флейта раздались.
Обрадован музыки громом,
Оставя чашку чаю с ромом,
Парис окружных городков,
Подходит к Ольге Петушков,
К Татьяне Ленский; Харликову,
Невесту переспелых лет,
Берет тамбовский мой поэт,
Умчал Буянов Пустякову,
И в залу высыпали все.
И бал блестит во всей красе.
\end{alltt}
\end{minipage}

\begin{minipage}[t]{\dimexpr 0.5\textwidth -\tabcolsep-.5pt}
\begin{alltt}\normalfont\centering
XL

В начале моего романа
(Смотрите первую тетрадь)
Хотелось вроде мне Альбана
Бал петербургский описать;
Но, развлечен пустым мечтаньем,
Я занялся воспоминаньем
О ножках мне знакомых дам.
По вашим узеньким следам,
О ножки, полно заблуждаться!
С изменой юности моей
Пора мне сделаться умней,
В делах и в слоге поправляться,
И эту пятую тетрадь
От отступлений очищать.
\end{alltt}
\end{minipage}
\clearpage

\begin{minipage}[t]{\dimexpr 0.5\textwidth -\tabcolsep-.5pt}
\begin{alltt}\normalfont\centering
XLI

Однообразный и безумный,
Как вихорь жизни молодой,
Кружится вальса вихорь шумный;
Чета мелькает за четой.
К минуте мщенья приближаясь,
Онегин, втайне усмехаясь,
Подходит к Ольге. Быстро с ней
Вертится около гостей,
Потом на стул ее сажает,
Заводит речь о том о сем;
Спустя минуты две потом
Вновь с нею вальс он продолжает;
Все в изумленье. Ленский сам
Не верит собственным глазам.
\end{alltt}
\end{minipage}

\begin{minipage}[t]{\dimexpr 0.5\textwidth -\tabcolsep-.5pt}
\begin{alltt}\normalfont\centering
XLII

Мазурка раздалась. Бывало,
Когда гремел мазурки гром,
В огромной зале все дрожало,
Паркет трещал под каблуком,
Тряслися, дребезжали рамы;
Теперь не то: и мы, как дамы,
Скользим по лаковым доскам.
Но в городах, по деревням
Еще мазурка сохранила
Первоначальные красы:
Припрыжки, каблуки, усы
Всё те же: их не изменила
Лихая мода, наш тиран,
Недуг новейших россиян.
\end{alltt}
\end{minipage}
\clearpage

\begin{minipage}[t]{\dimexpr 0.5\textwidth -\tabcolsep-.5pt}
\begin{alltt}\normalfont\centering
XLIII. XLIV

Буянов, братец мой задорный,
К герою нашему подвел
Татьяну с Ольгою; проворно
Онегин с Ольгою пошел;
Ведет ее, скользя небрежно,
И, наклонясь, ей шепчет нежно
Какой-то пошлый мадригал,
И руку жмет — и запылал
В ее лице самолюбивом
Румянец ярче. Ленский мой
Все видел: вспыхнул, сам не свой;
В негодовании ревнивом
Поэт конца мазурки ждет
И в котильон ее зовет.
\end{alltt}
\end{minipage}

\begin{minipage}[t]{\dimexpr 0.5\textwidth -\tabcolsep-.5pt}
\begin{alltt}\normalfont\centering
XLV

Но ей нельзя. Нельзя? Но что же?
Да Ольга слово уж дала
Онегину. О боже, боже!
Что слышит он? Она могла...
Возможно ль? Чуть лишь из пеленок,
Кокетка, ветреный ребенок!
Уж хитрость ведает она,
Уж изменять научена!
Не в силах Ленский снесть удара;
Проказы женские кляня,
Выходит, требует коня
И скачет. Пистолетов пара,
Две пули — больше ничего —
Вдруг разрешат судьбу его.
\end{alltt}
\end{minipage}
\clearpage
