\begin{minipage}[t]{\dimexpr 0.5\textwidth -\tabcolsep-.5pt}
\begin{alltt}\normalfont\centering
I. II. III. IV. V. VI.
VII
Чем меньше женщину мы любим,
Тем легче нравимся мы ей
И тем ее вернее губим
Средь обольстительных сетей.
Разврат, бывало, хладнокровный
Наукой славился любовной,
Сам о себе везде трубя
И наслаждаясь не любя.
Но эта важная забава
Достойна старых обезьян
Хваленых дедовских времян:
Ловласов обветшала слава
Со славой красных каблуков
И величавых париков.
\end{alltt}
\end{minipage}

\begin{minipage}[t]{\dimexpr 0.5\textwidth -\tabcolsep-.5pt}
\begin{alltt}\normalfont\centering
VIII

Кому не скучно лицемерить,
Различно повторять одно,
Стараться важно в том уверить,
В чем все уверены давно,
Всё те же слышать возраженья,
Уничтожать предрассужденья,
Которых не было и нет
У девочки в тринадцать лет!
Кого не утомят угрозы,
Моленья, клятвы, мнимый страх,
Записки на шести листах,
Обманы, сплетни, кольцы, слезы,
Надзоры теток, матерей
И дружба тяжкая мужей!
\end{alltt}
\end{minipage}
\clearpage

\begin{minipage}[t]{\dimexpr 0.5\textwidth -\tabcolsep-.5pt}
\begin{alltt}\normalfont\centering
IX

Так точно думал мой Евгений.
Он в первой юности своей
Был жертвой бурных заблуждений
И необузданных страстей.
Привычкой жизни избалован,
Одним на время очарован,
Разочарованный другим,
Желаньем медленно томим,
Томим и ветреным успехом,
Внимая в шуме и в тиши
Роптанье вечное души,
Зевоту подавляя смехом:
Вот как убил он восемь лет,
Утратя жизни лучший цвет.
\end{alltt}
\end{minipage}

\begin{minipage}[t]{\dimexpr 0.5\textwidth -\tabcolsep-.5pt}
\begin{alltt}\normalfont\centering
X

В красавиц он уж не влюблялся,
А волочился как-нибудь;
Откажут — мигом утешался;
Изменят — рад был отдохнуть.
Он их искал без упоенья,
А оставлял без сожаленья,
Чуть помня их любовь и злость.
Так точно равнодушный гость
На вист вечерний приезжает,
Садится; кончилась игра:
Он уезжает со двора,
Спокойно дома засыпает
И сам не знает поутру,
Куда поедет ввечеру.
\end{alltt}
\end{minipage}
\clearpage

\begin{minipage}[t]{\dimexpr 0.5\textwidth -\tabcolsep-.5pt}
\begin{alltt}\normalfont\centering
XI

Но, получив посланье Тани,
Онегин живо тронут был:
Язык девических мечтаний
В нем думы роем возмутил;
И вспомнил он Татьяны милой
И бледный цвет и вид унылый;
И в сладостный, безгрешный сон
Душою погрузился он.
Быть может, чувствий пыл старинный
Им на минуту овладел;
Но обмануть он не хотел
Доверчивость души невинной.
Теперь мы в сад перелетим,
Где встретилась Татьяна с ним.
\end{alltt}
\end{minipage}

\begin{minipage}[t]{\dimexpr 0.5\textwidth -\tabcolsep-.5pt}
\begin{alltt}\normalfont\centering
XII

Минуты две они молчали,
Но к ней Онегин подошел
И молвил: «Вы ко мне писали,
Не отпирайтесь. Я прочел
Души доверчивой признанья,
Любви невинной излиянья;
Мне ваша искренность мила;
Она в волненье привела
Давно умолкнувшие чувства;
Но вас хвалить я не хочу;
Я за нее вам отплачу
Признаньем также без искусства;
Примите исповедь мою:
Себя на суд вам отдаю.
\end{alltt}
\end{minipage}
\clearpage

\begin{minipage}[t]{\dimexpr 0.5\textwidth -\tabcolsep-.5pt}
\begin{alltt}\normalfont\centering
XIII

Когда бы жизнь домашним кругом
Я ограничить захотел;
Когда б мне быть отцом, супругом
Приятный жребий повелел;
Когда б семейственной картиной
Пленился я хоть миг единый, —
То, верно б, кроме вас одной
Невесты не искал иной.
Скажу без блесток мадригальных:
Нашед мой прежний идеал,
Я, верно б, вас одну избрал
В подруги дней моих печальных,
Всего прекрасного в залог,
И был бы счастлив... сколько мог!
\end{alltt}
\end{minipage}

\begin{minipage}[t]{\dimexpr 0.5\textwidth -\tabcolsep-.5pt}
\begin{alltt}\normalfont\centering
XIV

Но я не создан для блаженства;
Ему чужда душа моя;
Напрасны ваши совершенства:
Их вовсе недостоин я.
Поверьте (совесть в том порукой),
Супружество нам будет мукой.
Я, сколько ни любил бы вас,
Привыкнув, разлюблю тотчас;
Начнете плакать: ваши слезы
Не тронут сердца моего,
А будут лишь бесить его.
Судите ж вы, какие розы
Нам заготовит Гименей
И, может быть, на много дней.
\end{alltt}
\end{minipage}
\clearpage

\begin{minipage}[t]{\dimexpr 0.5\textwidth -\tabcolsep-.5pt}
\begin{alltt}\normalfont\centering
XV

Что может быть на свете хуже
Семьи, где бедная жена
Грустит о недостойном муже,
И днем и вечером одна;
Где скучный муж, ей цену зная
(Судьбу, однако ж, проклиная),
Всегда нахмурен, молчалив,
Сердит и холодно-ревнив!
Таков я. И того ль искали
Вы чистой, пламенной душой,
Когда с такою простотой,
С таким умом ко мне писали?
Ужели жребий вам такой
Назначен строгою судьбой?
\end{alltt}
\end{minipage}

\begin{minipage}[t]{\dimexpr 0.5\textwidth -\tabcolsep-.5pt}
\begin{alltt}\normalfont\centering
XVI

Мечтам и годам нет возврата;
Не обновлю души моей...
Я вас люблю любовью брата
И, может быть, еще нежней.
Послушайте ж меня без гнева:
Сменит не раз младая дева
Мечтами легкие мечты;
Так деревцо свои листы
Меняет с каждою весною.
Так видно небом суждено.
Полюбите вы снова: но...
Учитесь властвовать собою;
Не всякий вас, как я, поймет;
К беде неопытность ведет».
\end{alltt}
\end{minipage}
\clearpage

\begin{minipage}[t]{\dimexpr 0.5\textwidth -\tabcolsep-.5pt}
\begin{alltt}\normalfont\centering
XVII

Так проповедовал Евгений.
Сквозь слез не видя ничего,
Едва дыша, без возражений,
Татьяна слушала его.
Он подал руку ей. Печально
(Как говорится, машинально)
Татьяна молча оперлась,
Головкой томною склонясь;
Пошли домой вкруг огорода;
Явились вместе, и никто
Не вздумал им пенять на то.
Имеет сельская свобода
Свои счастливые права,
Как и надменная Москва.
\end{alltt}
\end{minipage}

\begin{minipage}[t]{\dimexpr 0.5\textwidth -\tabcolsep-.5pt}
\begin{alltt}\normalfont\centering
XVIII

Вы согласитесь, мой читатель,
Что очень мило поступил
С печальной Таней наш приятель;
Не в первый раз он тут явил
Души прямое благородство,
Хотя людей недоброхотство
В нем не щадило ничего:
Враги его, друзья его
(Что, может быть, одно и то же)
Его честили так и сяк.
Врагов имеет в мире всяк,
Но от друзей спаси нас, боже!
Уж эти мне друзья, друзья!
Об них недаром вспомнил я.
\end{alltt}
\end{minipage}
\clearpage

\begin{minipage}[t]{\dimexpr 0.5\textwidth -\tabcolsep-.5pt}
\begin{alltt}\normalfont\centering
XIX

А что? Да так. Я усыпляю
Пустые, черные мечты;
Я только в скобках замечаю,
Что нет презренной клеветы,
На чердаке вралем рожденной
И светской чернью ободренной,
Что нет нелепицы такой,
Ни эпиграммы площадной,
Которой бы ваш друг с улыбкой,
В кругу порядочных людей,
Без всякой злобы и затей,
Не повторил стократ ошибкой;
А впрочем, он за вас горой:
Он вас так любит... как родной!
\end{alltt}
\end{minipage}

\begin{minipage}[t]{\dimexpr 0.5\textwidth -\tabcolsep-.5pt}
\begin{alltt}\normalfont\centering
XX

Гм! гм! Читатель благородный,
Здорова ль ваша вся родня?
Позвольте: может быть, угодно
Теперь узнать вам от меня,
Что значит именно родные.
Родные люди вот какие:
Мы их обязаны ласкать,
Любить, душевно уважать
И, по обычаю народа,
О рождестве их навещать
Или по почте поздравлять,
Чтоб остальное время года
Не думали о нас они...
Итак, дай бог им долги дни!
\end{alltt}
\end{minipage}
\clearpage

\begin{minipage}[t]{\dimexpr 0.5\textwidth -\tabcolsep-.5pt}
\begin{alltt}\normalfont\centering
XXI

Зато любовь красавиц нежных
Надежней дружбы и родства:
Над нею и средь бурь мятежных
Вы сохраняете права.
Конечно так. Но вихорь моды,
Но своенравие природы,
Но мненья светского поток...
А милый пол, как пух, легок.
К тому ж и мнения супруга
Для добродетельной жены
Всегда почтенны быть должны;
Так ваша верная подруга
Бывает вмиг увлечена:
Любовью шутит сатана.
\end{alltt}
\end{minipage}

\begin{minipage}[t]{\dimexpr 0.5\textwidth -\tabcolsep-.5pt}
\begin{alltt}\normalfont\centering
XXII

Кого ж любить? Кому же верить?
Кто не изменит нам один?
Кто все дела, все речи мерит
Услужливо на наш аршин?
Кто клеветы про нас не сеет?
Кто нас заботливо лелеет?
Кому порок наш не беда?
Кто не наскучит никогда?
Призрака суетный искатель,
Трудов напрасно не губя,
Любите самого себя,
Достопочтенный мой читатель!
Предмет достойный: ничего
Любезней, верно, нет его.
\end{alltt}
\end{minipage}
\clearpage

\begin{minipage}[t]{\dimexpr 0.5\textwidth -\tabcolsep-.5pt}
\begin{alltt}\normalfont\centering
XXIII

Что было следствием свиданья?
Увы, не трудно угадать!
Любви безумные страданья
Не перестали волновать
Младой души, печали жадной;
Нет, пуще страстью безотрадной
Татьяна бедная горит;
Ее постели сон бежит;
Здоровье, жизни цвет и сладость,
Улыбка, девственный покой,
Пропало все, что звук пустой,
И меркнет милой Тани младость:
Так одевает бури тень
Едва рождающийся день.
\end{alltt}
\end{minipage}

\begin{minipage}[t]{\dimexpr 0.5\textwidth -\tabcolsep-.5pt}
\begin{alltt}\normalfont\centering
XXIV

Увы, Татьяна увядает,
Бледнеет, гаснет и молчит!
Ничто ее не занимает,
Ее души не шевелит.
Качая важно головою,
Соседи шепчут меж собою:
Пора, пора бы замуж ей!..
Но полно. Надо мне скорей
Развеселить воображенье
Картиной счастливой любви.
Невольно, милые мои,
Меня стесняет сожаленье;
Простите мне: я так люблю
Татьяну милую мою!
\end{alltt}
\end{minipage}
\clearpage

\begin{minipage}[t]{\dimexpr 0.5\textwidth -\tabcolsep-.5pt}
\begin{alltt}\normalfont\centering
XXV

Час от часу плененный боле
Красами Ольги молодой,
Владимир сладостной неволе
Предался полною душой.
Он вечно с ней. В ее покое
Они сидят в потемках двое;
Они в саду, рука с рукой,
Гуляют утренней порой;
И что ж? Любовью упоенный,
В смятенье нежного стыда,
Он только смеет иногда,
Улыбкой Ольги ободренный,
Развитым локоном играть
Иль край одежды целовать.
\end{alltt}
\end{minipage}

\begin{minipage}[t]{\dimexpr 0.5\textwidth -\tabcolsep-.5pt}
\begin{alltt}\normalfont\centering
XXVI

Он иногда читает Оле
Нравоучительный роман,
В котором автор знает боле
Природу, чем Шатобриан,
А между тем две, три страницы
(Пустые бредни, небылицы,
Опасные для сердца дев)
Он пропускает, покраснев.
Уединясь от всех далеко,
Они над шахматной доской,
На стол облокотясь, порой
Сидят, задумавшись глубоко,
И Ленский пешкою ладью
Берет в рассеянье свою.
\end{alltt}
\end{minipage}
\clearpage

\begin{minipage}[t]{\dimexpr 0.5\textwidth -\tabcolsep-.5pt}
\begin{alltt}\normalfont\centering
XXVII

Поедет ли домой, и дома
Он занят Ольгою своей.
Летучие листки альбома
Прилежно украшает ей:
То в них рисует сельски виды,
Надгробный камень, храм Киприды,
Или на лире голубка
Пером и красками слегка;
То на листках воспоминанья
Пониже подписи других
Он оставляет нежный стих,
Безмолвный памятник мечтанья,
Мгновенной думы долгий след,
Все тот же после многих лет.
\end{alltt}
\end{minipage}

\begin{minipage}[t]{\dimexpr 0.5\textwidth -\tabcolsep-.5pt}
\begin{alltt}\normalfont\centering
XXVIII

Конечно, вы не раз видали
Уездной барышни альбом,
Что все подружки измарали
С конца, с начала и кругом.
Сюда, назло правописанью,
Стихи без меры, по преданью
В знак дружбы верной внесены,
Уменьшены, продолжены.
На первом листике встречаешь
Qu'écrirez-vous sur ces tablettes,
И подпись: t. à v. Annette;
А на последнем прочитаешь:
«Кто любит более тебя,
Пусть пишет далее меня».
\end{alltt}
\end{minipage}
\clearpage

\begin{minipage}[t]{\dimexpr 0.5\textwidth -\tabcolsep-.5pt}
\begin{alltt}\normalfont\centering
XXIX

Тут непременно вы найдете
Два сердца, факел и цветки;
Тут верно клятвы вы прочтете
В любви до гробовой доски;
Какой-нибудь пиит армейский
Тут подмахнул стишок злодейский.
В такой альбом, мои друзья,
Признаться, рад писать и я,
Уверен будучи душою,
Что всякий мой усердный вздор
Заслужит благосклонный взор
И что потом с улыбкой злою
Не станут важно разбирать,
Остро иль нет я мог соврать.
\end{alltt}
\end{minipage}

\begin{minipage}[t]{\dimexpr 0.5\textwidth -\tabcolsep-.5pt}
\begin{alltt}\normalfont\centering
XXX

Но вы, разрозненные томы
Из библиотеки чертей,
Великолепные альбомы,
Мученье модных рифмачей,
Вы, украшенные проворно
Толстого кистью чудотворной
Иль Баратынского пером,
Пускай сожжет вас божий гром!
Когда блистательная дама
Мне свой in-quarto подает,
И дрожь и злость меня берет,
И шевелится эпиграмма
Во глубине моей души,
А мадригалы им пиши!
\end{alltt}
\end{minipage}
\clearpage

\begin{minipage}[t]{\dimexpr 0.5\textwidth -\tabcolsep-.5pt}
\begin{alltt}\normalfont\centering
XXXI

Не мадригалы Ленский пишет
В альбоме Ольги молодой;
Его перо любовью дышит,
Не хладно блещет остротой;
Что ни заметит, ни услышит
Об Ольге, он про то и пишет:
И, полны истины живой,
Текут элегии рекой.
Так ты, Языков вдохновенный,
В порывах сердца своего,
Поешь бог ведает кого,
И свод элегий драгоценный
Представит некогда тебе
Всю повесть о твоей судьбе.
\end{alltt}
\end{minipage}

\begin{minipage}[t]{\dimexpr 0.5\textwidth -\tabcolsep-.5pt}
\begin{alltt}\normalfont\centering
XXXII

Но тише! Слышишь? Критик строгий
Повелевает сбросить нам
Элегии венок убогий,
И нашей братье рифмачам
Кричит: «Да перестаньте плакать,
И всё одно и то же квакать,
Жалеть о прежнем, о былом:
Довольно, пойте о другом!»
— Ты прав, и верно нам укажешь
Трубу, личину и кинжал,
И мыслей мертвый капитал
Отвсюду воскресить прикажешь:
Не так ли, друг? — Ничуть. Куда!
«Пишите оды, господа,
\end{alltt}
\end{minipage}
\clearpage

\begin{minipage}[t]{\dimexpr 0.5\textwidth -\tabcolsep-.5pt}
\begin{alltt}\normalfont\centering
XXXIII

Как их писали в мощны годы,
Как было встарь заведено...»
— Одни торжественные оды!
И, полно, друг; не все ль равно?
Припомни, что сказал сатирик!
«Чужого толка» хитрый лирик
Ужели для тебя сносней
Унылых наших рифмачей? —
«Но всё в элегии ничтожно;
Пустая цель ее жалка;
Меж тем цель оды высока
И благородна...» Тут бы можно
Поспорить нам, но я молчу:
Два века ссорить не хочу.
\end{alltt}
\end{minipage}

\begin{minipage}[t]{\dimexpr 0.5\textwidth -\tabcolsep-.5pt}
\begin{alltt}\normalfont\centering
XXXIV

Поклонник славы и свободы,
В волненье бурных дум своих,
Владимир и писал бы оды,
Да Ольга не читала их.
Случалось ли поэтам слезным
Читать в глаза своим любезным
Свои творенья? Говорят,
Что в мире выше нет наград.
И впрям, блажен любовник скромный,
Читающий мечты свои
Предмету песен и любви,
Красавице приятно-томной!
Блажен... хоть, может быть, она
Совсем иным развлечена.
\end{alltt}
\end{minipage}
\clearpage

\begin{minipage}[t]{\dimexpr 0.5\textwidth -\tabcolsep-.5pt}
\begin{alltt}\normalfont\centering
XXXV

Но я плоды моих мечтаний
И гармонических затей
Читаю только старой няне,
Подруге юности моей,
Да после скучного обеда
Ко мне забредшего соседа,
Поймав нежданно за полу,
Душу трагедией в углу,
Или (но это кроме шуток),
Тоской и рифмами томим,
Бродя над озером моим,
Пугаю стадо диких уток:
Вняв пенью сладкозвучных строф,
Они слетают с берегов.
\end{alltt}
\end{minipage}

\begin{minipage}[t]{\dimexpr 0.5\textwidth -\tabcolsep-.5pt}
\begin{alltt}\normalfont\centering
XXXVI. XXXVII

А что ж Онегин? Кстати, братья!
Терпенья вашего прошу:
Его вседневные занятья
Я вам подробно опишу.
Онегин жил анахоретом:
В седьмом часу вставал он летом
И отправлялся налегке
К бегущей под горой реке;
Певцу Гюльнары подражая,
Сей Геллеспонт переплывал,
Потом свой кофе выпивал,
Плохой журнал перебирая,
И одевался...
\end{alltt}
\end{minipage}
\clearpage

\begin{minipage}[t]{\dimexpr 0.5\textwidth -\tabcolsep-.5pt}
\begin{alltt}\normalfont\centering
XXXVIII. XXXIX

Прогулки, чтенье, сон глубокой,
Лесная тень, журчанье струй,
Порой белянки черноокой
Младой и свежий поцелуй,
Узде послушный конь ретивый,
Обед довольно прихотливый,
Бутылка светлого вина,
Уединенье, тишина:
Вот жизнь Онегина святая;
И нечувствительно он ей
Предался, красных летних дней
В беспечной неге не считая,
Забыв и город, и друзей,
И скуку праздничных затей.
\end{alltt}
\end{minipage}

\begin{minipage}[t]{\dimexpr 0.5\textwidth -\tabcolsep-.5pt}
\begin{alltt}\normalfont\centering
XL

Но наше северное лето,
Карикатура южных зим,
Мелькнет и нет: известно это,
Хоть мы признаться не хотим.
Уж небо осенью дышало,
Уж реже солнышко блистало,
Короче становился день,
Лесов таинственная сень
С печальным шумом обнажалась,
Ложился на поля туман,
Гусей крикливых караван
Тянулся к югу: приближалась
Довольно скучная пора;
Стоял ноябрь уж у двора.
\end{alltt}
\end{minipage}
\clearpage

\begin{minipage}[t]{\dimexpr 0.5\textwidth -\tabcolsep-.5pt}
\begin{alltt}\normalfont\centering
XLI

Встает заря во мгле холодной;
На нивах шум работ умолк;
С своей волчихою голодной
Выходит на дорогу волк;
Его почуя, конь дорожный
Храпит — и путник осторожный
Несется в гору во весь дух;
На утренней заре пастух
Не гонит уж коров из хлева,
И в час полуденный в кружок
Их не зовет его рожок;
В избушке распевая, дева 23
Прядет, и, зимних друг ночей,
Трещит лучинка перед ней.
\end{alltt}
\end{minipage}

\begin{minipage}[t]{\dimexpr 0.5\textwidth -\tabcolsep-.5pt}
\begin{alltt}\normalfont\centering
XLII

И вот уже трещат морозы
И серебрятся средь полей...
(Читатель ждет уж рифмы розы;
На, вот возьми ее скорей!)
Опрятней модного паркета
Блистает речка, льдом одета.
Мальчишек радостный народ 24
Коньками звучно режет лед;
На красных лапках гусь тяжелый,
Задумав плыть по лону вод,
Ступает бережно на лед,
Скользит и падает; веселый
Мелькает, вьется первый снег,
Звездами падая на брег.
\end{alltt}
\end{minipage}
\clearpage

\begin{minipage}[t]{\dimexpr 0.5\textwidth -\tabcolsep-.5pt}
\begin{alltt}\normalfont\centering
XLIII

В глуши что делать в эту пору?
Гулять? Деревня той порой
Невольно докучает взору
Однообразной наготой.
Скакать верхом в степи суровой?
Но конь, притупленной подковой
Неверный зацепляя лед,
Того и жди, что упадет.
Сиди под кровлею пустынной,
Читай: вот Прадт, вот W. Scott.
Не хочешь? — поверяй расход,
Сердись иль пей, и вечер длинный
Кой-как пройдет, а завтра тож,
И славно зиму проведешь.
\end{alltt}
\end{minipage}

\begin{minipage}[t]{\dimexpr 0.5\textwidth -\tabcolsep-.5pt}
\begin{alltt}\normalfont\centering
XLIV

Прямым Онегин Чильд-Гарольдом
Вдался в задумчивую лень:
Со сна садится в ванну со льдом,
И после, дома целый день,
Один, в расчеты погруженный,
Тупым кием вооруженный,
Он на бильярде в два шара
Играет с самого утра.
Настанет вечер деревенский:
Бильярд оставлен, кий забыт,
Перед камином стол накрыт,
Евгений ждет: вот едет Ленский
На тройке чалых лошадей;
Давай обедать поскорей!
\end{alltt}
\end{minipage}
\clearpage

\begin{minipage}[t]{\dimexpr 0.5\textwidth -\tabcolsep-.5pt}
\begin{alltt}\normalfont\centering
XLV

Вдовы Клико или Моэта
Благословенное вино
В бутылке мерзлой для поэта
На стол тотчас принесено.
Оно сверкает Ипокреной; 25
Оно своей игрой и пеной
(Подобием того-сего)
Меня пленяло: за него
Последний бедный лепт, бывало,
Давал я. Помните ль, друзья?
Его волшебная струя
Рождала глупостей не мало,
А сколько шуток и стихов,
И споров, и веселых снов!
\end{alltt}
\end{minipage}

\begin{minipage}[t]{\dimexpr 0.5\textwidth -\tabcolsep-.5pt}
\begin{alltt}\normalfont\centering
XLVI

Но изменяет пеной шумной
Оно желудку моему,
И я Бордо благоразумный
Уж нынче предпочел ему.
К Аи я больше не способен;
Аи любовнице подобен
Блестящей, ветреной, живой,
И своенравной, и пустой...
Но ты, Бордо, подобен другу,
Который, в горе и в беде,
Товарищ завсегда, везде,
Готов нам оказать услугу
Иль тихий разделить досуг.
Да здравствует Бордо, наш друг!
\end{alltt}
\end{minipage}
\clearpage

\begin{minipage}[t]{\dimexpr 0.5\textwidth -\tabcolsep-.5pt}
\begin{alltt}\normalfont\centering
XLVII

Огонь потух; едва золою
Подернут уголь золотой;
Едва заметною струею
Виется пар, и теплотой
Камин чуть дышит. Дым из трубок
В трубу уходит. Светлый кубок
Еще шипит среди стола.
Вечерняя находит мгла...
(Люблю я дружеские враки
И дружеский бокал вина
Порою той, что названа
Пора меж волка и собаки,
А почему, не вижу я.)
Теперь беседуют друзья:
\end{alltt}
\end{minipage}

\begin{minipage}[t]{\dimexpr 0.5\textwidth -\tabcolsep-.5pt}
\begin{alltt}\normalfont\centering
XLVIII

«Ну, что соседки? Что Татьяна?
Что Ольга резвая твоя?»
— Налей еще мне полстакана...
Довольно, милый... Вся семья
Здорова; кланяться велели.
Ах, милый, как похорошели
У Ольги плечи, что за грудь!
Что за душа!... Когда-нибудь
Заедем к ним; ты их обяжешь;
А то, мой друг, суди ты сам:
Два раза заглянул, а там
Уж к ним и носу не покажешь.
Да вот... какой же я болван!
Ты к ним на той неделе зван.
\end{alltt}
\end{minipage}
\clearpage

\begin{minipage}[t]{\dimexpr 0.5\textwidth -\tabcolsep-.5pt}
\begin{alltt}\normalfont\centering
XLIX

«Я?» — Да, Татьяны именины
В субботу. Оленька и мать
Велели звать, и нет причины
Тебе на зов не приезжать. —
«Но куча будет там народу
И всякого такого сброду...»
— И, никого, уверен я!
Кто будет там? своя семья.
Поедем, сделай одолженье!
Ну, что ж? — «Согласен». — Как ты мил! —
При сих словах он осушил
Стакан, соседке приношенье,
Потом разговорился вновь
Про Ольгу: такова любовь!
\end{alltt}
\end{minipage}

\begin{minipage}[t]{\dimexpr 0.5\textwidth -\tabcolsep-.5pt}
\begin{alltt}\normalfont\centering
L

Он весел был. Чрез две недели
Назначен был счастливый срок.
И тайна брачныя постели,
И сладостной любви венок
Его восторгов ожидали.
Гимена хлопоты, печали,
Зевоты хладная чреда
Ему не снились никогда.
Меж тем как мы, враги Гимена,
В домашней жизни зрим один
Ряд утомительных картин,
Роман во вкусе Лафонтена... 26
Мой бедный Ленский, сердцем он
Для оной жизни был рожден.
\end{alltt}
\end{minipage}
\clearpage

\begin{minipage}[t]{\dimexpr 0.5\textwidth -\tabcolsep-.5pt}
\begin{alltt}\normalfont\centering
LI

Он был любим... по крайней мере
Так думал он, и был счастлив.
Стократ блажен, кто предан вере,
Кто, хладный ум угомонив,
Покоится в сердечной неге,
Как пьяный путник на ночлеге,
Или, нежней, как мотылек,
В весенний впившийся цветок;
Но жалок тот, кто все предвидит,
Чья не кружится голова,
Кто все движенья, все слова
В их переводе ненавидит,
Чье сердце опыт остудил
И забываться запретил!
\end{alltt}
\end{minipage}
\clearpage
